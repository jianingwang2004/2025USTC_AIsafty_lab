%%
%% This is file `sample-acmtog.tex',
%% generated with the docstrip utility.
%%
%% The original source files were:
%%
%% samples.dtx  (with options: `acmtog')
%% 
%% IMPORTANT NOTICE:
%% 
%% For the copyright see the source file.
%% 
%% Any modified versions of this file must be renamed
%% with new filenames distinct from sample-acmtog.tex.
%% 
%% For distribution of the original source see the terms
%% for copying and modification in the file samples.dtx.
%% 
%% This generated file may be distributed as long as the
%% original source files, as listed above, are part of the
%% same distribution. (The sources need not necessarily be
%% in the same archive or directory.)
%%
%% The first command in your LaTeX source must be the \documentclass command.
\documentclass[acmtog]{ctexart}

%%
%% \BibTeX command to typeset BibTeX logo in the docs
\AtBeginDocument{%
  \providecommand\BibTeX{{%
    \normalfont B\kern-0.5em{\scshape i\kern-0.25em b}\kern-0.8em\TeX}}}

%% Rights management information.  This information is sent to you
%% when you complete the rights form.  These commands have SAMPLE
%% values in them; it is your responsibility as an author to replace
%% the commands and values with those provided to you when you
%% complete the rights form.


%%
%% These commands are for a JOURNAL article.

%%
%% Submission ID.
%% Use this when submitting an article to a sponsored event. You'll
%% receive a unique submission ID from the organizers
%% of the event, and this ID should be used as the parameter to this command.
%%\acmSubmissionID{123-A56-BU3}

%%
%% The majority of ACM publications use numbered citations and
%% references.  The command \citestyle{authoryear} switches to the
%% "author year" style.
%%
%% If you are preparing content for an event
%% sponsored by ACM SIGGRAPH, you must use the "author year" style of
%% citations and references.

%%
%% end of the preamble, start of the body of the document source.
\usepackage{booktabs} 
\begin{document}

%%
%% The "title" command has an optional parameter,
%% allowing the author to define a "short title" to be used in page headers.
\title{面向自动驾驶系统的图像分类模型对抗攻击与防御技术研究综述}

%%
%% The "author" command and its associated commands are used to define
%% the authors and their affiliations.
%% Of note is the shared affiliation of the first two authors, and the
%% "authornote" and "authornotemark" commands
%% used to denote shared contribution to the research.
\author{PB22051022王嘉宁}
%%
%% By default, the full list of authors will be used in the page
%% headers. Often, this list is too long, and will overlap
%% other information printed in the page headers. This command allows
%% the author to define a more concise list
%% of authors' names for this purpose.

%%
%% The abstract is a short summary of the work to be presented in the
%% article.
\begin{abstract}
  随着深度学习技术在自动驾驶视觉感知系统中的广泛应用,对抗攻击及其防御策略已成为保障智能交通安全的核心研究课题。本综述系统梳理了2020-2025年间35篇代表性研究成果,全面分析了自动驾驶场景下图像分类模型面临的对抗威胁及其防护机制。研究表明,对抗样本通过微小扰动即可导致模型误判,而物理世界的补丁攻击对交通标志识别构成严峻威胁。在防御技术方面,融合对抗训练的主动防御、基于输入重构的被动防御以及针对物理攻击的专用防护构成了多层次防御体系。本文深入探讨了攻击与防御技术在动态环境中的对抗效应,总结了实验评估指标与性能对比,并指出物理攻击防御、轻量化防护架构、多模态融合防御、对抗样本可解释性及标准化测试框架五大未来研究方向。该综述为提升自动驾驶系统的对抗鲁棒性提供了理论支持和技术参考。
\end{abstract}

%%
%% The code below is generated by the tool at http://dl.acm.org/ccs.cfm.
%% Please copy and paste the code instead of the example below.
%%



%%
%% This command processes the author and affiliation and title
%% information and builds the first part of the formatted document.
\maketitle

\section{引言}
自动驾驶技术的快速发展正深刻改变着交通运输行业的格局,而基于深度学习的视觉感知系统作为自动驾驶的“眼睛”,承担着环境感知、目标检测和决策支持等关键任务。然而,近年来的研究发现,深度学习模型在面对精心设计的扰动时表现出惊人的脆弱性,这类扰动生成的样本称为对抗样本。在自动驾驶场景中,对抗攻击可能导致交通标志误识别、障碍物漏检等严重后果,威胁行车安全。例如,唐军等人研究表明,通过对交通标志添加人眼难以察觉的扰动,可使自动驾驶系统的识别准确率从97.69\%骤降至4.82\%。

对抗攻击的概念最初由Szegedy等人在2014年提出,其研究表明深度神经网络容易受到人为设计的微小扰动干扰,导致模型以高置信度输出错误结果[citation:34]。这种现象在自动驾驶等安全关键领域尤其值得关注,因为攻击者可能通过篡改交通标志或道路标记,诱导车辆做出错误反应,进而引发交通事故。王文萱等人在综述中指出,对抗攻击根据应用场景可分为数字域攻击和物理域攻击两大类,前者直接在数字图像上修改像素值,后者则通过扰动物理实体或传感器间接影响模型输入。

本文聚焦于自动驾驶系统中的图像分类模型对抗攻防技术,系统梳理了该领域的最新研究进展。综述结构如下:第二部分深入分析对抗攻击技术,包括数字域和物理域攻击方法;第三部分系统分类防御技术;第四部分探讨自动驾驶场景的特殊挑战与解决方案;第五部分总结实验评估方法;最后展望未来研究方向。通过多维度分析,旨在为提升自动驾驶系统的安全性和鲁棒性提供理论支持和技术参考。

\section{对抗攻击技术研究进展}

\subsection{攻击分类与特征}

自动驾驶系统中的对抗攻击可根据攻击环境、攻击目标和攻击知识三个维度进行分类。根据攻击环境,可分为数字域攻击和物理域攻击。数字域攻击直接在图像像素层面添加扰动,而物理域攻击则通过在真实物体表面添加扰动(如贴纸、涂鸦)实现攻击。根据攻击目标,可分为有目标攻击和无目标攻击,前者诱使模型输出特定错误类别,后者仅需导致模型错误输出即可。根据攻击知识,可分为白盒攻击(攻击者完全了解模型结构及参数)、黑盒攻击(攻击者仅能查询模型输出)和灰盒攻击(攻击者了解部分模型信息)。

\begin{table}
  \caption{对抗攻击分类及典型特征}
  \label{tab:freq}
  \begin{tabular}{ccl}
    \toprule
    攻击类型&典型特征&自动驾驶场景案例\\
    \midrule
    数字域攻击&像素级扰动,易于实施&篡改车载摄像头输入的交通标志图像\\
    物理域攻击&需考虑环境因素,鲁棒性强&在真实交通标志上粘贴对抗补丁\\
    有目标攻击&诱导特定错误输出&将“停车”标志识别为“限速60”\\
    无目标攻击&仅需导致错误识别&导致任何交通标志误分类\\
    白盒攻击&攻击效率高,威胁来自内部&针对已知模型架构的精确攻击\\
    黑盒攻击&依赖迁移性,更具现实威胁&对未知商用自动驾驶系统的攻击\\
  \bottomrule
\end{tabular}
\end{table}

对抗攻击在自动驾驶场景中表现出三大核心特征:隐蔽性、迁移性和物理可实现性。隐蔽性指扰动在人类视觉难以察觉的范围内;迁移性指针对一个模型生成的对抗样本对其他模型同样有效;物理可实现性则强调扰动在真实物理环境中依然有效[citation:14]。韩家宝等人的研究表明,攻击者可通过物理手段在路标上添加精心设计的涂鸦,导致自动驾驶车辆在80米外就开始错误识别交通标志

\subsection{数字域攻击方法}

数字域攻击通过在数字图像中添加微小扰动实现攻击,是自动驾驶系统面临的基础性威胁。快速梯度符号法(FGSM)是最早提出的攻击方法之一,它利用模型梯度的符号信息生成扰动,计算效率高,但攻击成功率有限1。为提升攻击效果,Kurakin等人提出基于迭代的BIM方法(BIM-a和BIM-b),通过多次小步长优化显著提高了攻击成功率1。

唐军等人在研究中采用了基于敏感性分析的噪声叠加攻击策略,针对典型的图像分类模型ResNet开展白盒对抗攻击1。该策略首先计算模型对输入图像的敏感度分布,然后在敏感度高的区域集中添加扰动,最大化攻击效果。实验表明,这种方法生成的对抗样本可使ResNet模型在交通标志识别任务中的准确率下降90\%以上1。

肖子勤等人综述中介绍了基于雅可比矩阵的显著图攻击(JSMA),该方法通过计算前向导数构造显著图,仅修改少量关键像素即可实现攻击7。张光华等人研究则揭示了JSMA攻击的另一面——其可用于去除深度神经网络后门,体现了攻防技术的辩证关系[citation:11]。

\subsection{物理域攻击与补丁攻击}

物理域攻击需要克服现实环境中的光线变化、视角偏移、天气干扰等挑战,技术难度更高。对抗补丁攻击是物理域攻击的主流形式,攻击者通过设计可打印的对抗性图案并将其附着在目标物体或背景上,误导模型识别。邓欢等人综述指出,物理补丁攻击需满足三大条件:视觉自然性、位置鲁棒性和尺度不变性4。

在自动驾驶场景中,物理攻击主要针对交通标志识别系统。翔云等人提出一种通用防御物理空间补丁对抗攻击方法,通过对抗训练增强模型对物理扰动的鲁棒性[citation:22]。该方法在训练过程中引入多种变换(如旋转、缩放、亮度调整),模拟物理环境变化,使模型学习忽略这些扰动。

陈国凯等人的研究表明,针对物理攻击的鲁棒性要求显著高于数字攻击,因为物理扰动必须考虑三维空间转换、光照变化和摄像机参数等多种因素10。他们设计了一种可贴在路标上的圆形对抗贴纸,实验显示即使在距离30米外、不同光照条件下,也能使自动驾驶系统的识别错误率达到75\%以上。

\subsection{迁移攻击与黑盒攻击}

迁移攻击是黑盒场景下的主要攻击手段,利用对抗样本的跨模型迁移性实现攻击。周栋等人针对空间视觉目标检测提出的对抗攻击算法,通过优化扰动的可迁移性,实现了对未知模型的攻击成功率提升15\%-20\%。其核心思想是打破对抗样本对特定模型的过拟合,增强泛化能力。

一种针对图像分类任务的对抗迁移攻击方法创新性地引入平移错位策略7。该方法首先对输入图像进行填补,然后基于预设平移错位策略对填充样本进行随机裁剪,得到增广样本图像。通过这种方式,提升了输入样本的多样性,避免生成的对抗样本过拟合源模型,从而有效攻击黑盒模型和防御模型。

刘文钊等人指出,对抗样本的迁移性源于不同模型在决策边界上的相似性,以及深度神经网络在高维空间中的线性特性5。通过动量迭代方法增强扰动的迁移性,攻击者可以生成对多种防御模型均有效的通用对抗样本。

\section{对抗防御技术体系}

\subsection{主动防御机制}

主动防御旨在通过增强模型自身鲁棒性,从根本上抵御对抗攻击。对抗训练是最广泛应用的主动防御策略,其核心思想是将对抗样本注入训练过程,使模型学习正确识别此类样本。王滨等人提出剪枝与鲁棒蒸馏融合方法,通过知识蒸馏将鲁棒性从大型模型迁移至轻量模型,解决了车载设备算力有限的问题[citation:24]。实验表明,该方法在CIFAR-10数据集上将轻量模型的鲁棒性提升了25\%,同时保持实时性要求。

孙家泽等人设计的陷阱式集成对抗防御网络(TrapNet)创新性地利用“可攻击空间假设”[citation:18]。该网络包含多个子模型,每个子模型设置不同的决策陷阱,诱使攻击者进入预设的无效攻击方向。当检测到对抗样本时,系统激活陷阱机制,将其导向安全输出。这种方法在交通标志识别任务中使多种主流攻击的成功率降低了30\%-60\%。

\begin{table}
  \caption{主动防御技术对比}
  \label{tab:freq}
  \begin{tabular}{ccl}
    \toprule
    防御方法&优势&局限性\\
    \midrule
    对抗训练&直接提升模型鲁棒性&训练复杂度高,可能降低标准准确率\\
    鲁棒蒸馏&实现轻量化防御,满足实时性&依赖教师模型质量\\
    陷阱网络&主动防御,高灵活性&需精确设计陷阱策略\\
    随机平滑&实现认证鲁棒性&推理延迟增加\\
  \bottomrule
\end{tabular}
\end{table}

\begin{table}
  \caption{续表}
  \label{tab:freq}
  \begin{tabular}{ccl}
    \toprule
    防御方法&核心技术&适用场景\\
    \midrule
    对抗训练&对抗样本注入训练集&高安全性要求的端到端系统\\
    鲁棒蒸馏&知识从鲁棒教师模型迁移&车载边缘计算设备\\
    陷阱网络&预设可攻击空间诱导攻击者&多模型协同决策系统\\
    随机平滑&输入随机化处理&对延迟不敏感的安全模块\\
  \bottomrule
\end{tabular}
\end{table}

王丹妮等人提出基于高斯增强和迭代攻击的对抗训练防御方法,在训练过程中引入高斯噪声增强和迭代攻击样本生成,显著提升了模型在复杂环境下的鲁棒性[citation:28]。该方法在Cityscapes自动驾驶数据集上的测试表明,模型对FGSM、PGD等主流攻击的防御成功率超过85\%。

\subsection{被动防御机制}

被动防御不改变模型本身,而是通过预处理输入数据或检测异常样本实现防护。范宇豪等人提出的基于插值法的防御算法(IDA)是典型代表[citation:33]。该算法在检测到模型识别率低于阈值时,对输入图像进行缩放和重构处理:缩小时使用基于像素区域算法,恢复原尺寸时采用立方卷积插值算法。实验证明,这种方法对FGSM攻击生成的对抗样本识别率从0.47\%提升至99.99\%,且不影响人类视觉识别。

张帅等人开发的自适应像素去噪方法通过分析图像局部统计特征,区分正常像素与对抗扰动像素[citation:20]。该方法在ImageNet数据集上将PGD攻击的成功率降低了70\%,且处理延迟仅增加15ms,满足自动驾驶实时性要求。

秦书晨等人系统总结了对抗样本检测技术,包括基于特征分析、基于不确定性量化和基于辅助检测网络的方法6。他们指出,在自动驾驶系统中,检测到对抗样本后应启动安全回退机制,如切换至冗余模型或提示人工接管,确保行车安全。

\subsection{物理攻击专用防御}

物理攻击防御需考虑环境动态性与传感器噪声。孙安临等人针对交通路标防御提出多视角融合验证方案,通过融合多个摄像头数据,降低单点攻击风险[citation:25]。该方案的核心思想是:单个摄像头可能被对抗补丁欺骗,但多个视角同时被欺骗的概率极低。系统通过比较不同视角的识别结果,检测并排除异常输出。

冼卓滢等人采用宽度学习系统(BLS)替代传统深度学习模型,构建轻量级防御框架[citation:16]。BLS无需深度结构,通过增量学习动态更新,在资源受限环境下仍保持高效。实验表明,该系统在物理对抗样本检测中达到89.6\%的准确率,且推理速度比传统深度学习模型快3倍。

葛佳伟等人分析指出,针对物理攻击的有效防御需结合数字防御与物理防护双重策略[citation:19]。数字防御指模型层面的鲁棒性提升,而物理防护则包括在交通标志表面添加防护涂层或特殊材料,使对抗贴纸难以附着,或安装物理屏障阻止攻击者接近关键交通设施。

\section{自动驾驶应用场景的特殊挑战}

\subsection{场景特性与威胁分析}

自动驾驶系统面临的环境具有高度动态性、不可预测性和安全关键性三大特征,使其对抗防御面临独特挑战。杨弋鋆等人研究表明,自动驾驶场景的对抗攻击可能导致多级连锁反应:当单个车辆模型受到攻击做出错误判断时,这些错误可能在整个交通系统中传播,引发一系列后续错误[citation:34]。例如,攻击单个车辆模型可能引发错误的交通决策,进而导致交通堵塞甚至事故。

刘佳玮等人分析了自动驾驶视觉系统面临的三重威胁模型:传感器层面攻击(干扰摄像头输入)、模型层面攻击(操纵深度学习模型)和系统层面攻击(破坏多传感器融合机制)[citation:26]。这些攻击可能单独实施,也可能协同进行,形成复杂攻击链。

在环境动态性方面,自动驾驶系统需应对光照变化、天气条件、运动模糊等挑战。郭敏等人的研究表明,雨雾天气会显著降低现有防御机制的效果,因为自然噪声与对抗扰动的叠加增加了辨识难度[citation:35]。他们提出结合天气自适应预处理与对抗训练的方法,在恶劣天气下仍保持较高防御性能。

\subsection{防御方案设计}

针对自动驾驶场景的特殊需求,研究者提出了多种专用防御方案。李前等人探讨了云边端协同防御架构,通过分布式计算实现高效防护[citation:23]。在该架构中,云端负责复杂模型训练和全局更新,边缘节点(路侧单元)负责区域威胁情报共享,车载端执行实时轻量级防御。这种分层架构既满足实时性要求,又能应对新型攻击。

孙安临等人专门研究了对交通路标对抗攻击的防御方案,提出一种结合空间变换与区域分割的方法[citation:25]。该方法首先检测图像中的交通标志区域,然后应用随机旋转和缩放等空间变换削弱扰动效果,最后使用分割网络提取标志特征进行分类。这种方案在真实道路测试中成功防御了多种物理补丁攻击。

陈卓等人针对联邦学习场景提出选择性防御策略,解决分布式训练中的对抗攻击问题[citation:17]。该方法通过分析参数更新贡献度识别恶意节点,动态调整聚合权重,保护车联网中的协同学习系统。实验显示,该策略在存在20\%恶意节点的情况下,仍能保持90\%以上的模型准确率。

\section{实验评估与性能对比}

\subsection{评估指标与数据集}

对抗攻防技术的评估需综合考虑防御效果、计算开销和模型通用性三大维度。常用评估指标包括:清洁样本准确率(衡量模型原始性能)、对抗样本准确率(评估防御效果)、推理延迟(测量计算开销)和跨数据集泛化能力(检验模型通用性)6。

自动驾驶领域常用数据集包括:
\begin{itemize}
  \item 无防御模型在FGSM攻击下识别率降至5\%以下
  \item 传统对抗训练可提升至60\%-70\%
  \item 其提出的插值法(IDA)联合对抗训练方案可提升至82\%以上
\end{itemize}



\subsection{主流方法性能对比}

唐军等人的实验表明,在自动驾驶场景中:
\begin{itemize}
  \item GTSRB:德国交通标志识别数据集,包含50多种交通标志,超过5万张图像
  \item LISA:美国交通标志数据集,涵盖47种标志类型
  \item Cityscapes:城市场景理解数据集,提供高分辨率街景图像
  \item BDD100K:大规模多样化驾驶数据集,包含10万段视频[citation:14][citation:25]
\end{itemize}

\begin{table}
  \caption{自动驾驶场景对抗防御性能对比}
  \label{tab:freq}
  \begin{tabular}{ccl}
    \toprule
    防御方法	&清洁样本准确率	&对抗样本准确率	&推理延迟增加	&适用环境
    \midrule
    无防御基准	& 97.69\%	& 4.82\%	& 0ms	& 实验室环境\\
    传统对抗训练	& 95.20\%	& 65.30\%	& +10ms	& 简单路况\\
    插值法(IDA)	& 96.50\%	& 82.28\%	& +15ms	& 中等动态环境\\
    多视角融合	& 96.80\%	& 88.50\%	& +20ms	& 复杂动态环境\\
    鲁棒蒸馏轻量防御	& 94.70\%	& 76.40\%	& +5ms	&资源受限设备\\
  \bottomrule
\end{tabular}
\end{table}


\section{未来研究方向}

\section{结论}

本综述系统分析了面向自动驾驶图像分类模型的对抗攻击与防御技术研究进展。研究表明,对抗攻击尤其是物理补丁攻击,已成为自动驾驶安全的重大威胁;而融合多层次防御策略(主动加固、被动检测、物理防护)的综合方案是提升系统鲁棒性的有效途径。

在攻击技术方面,数字域攻击已发展出多种高效生成方法,物理域攻击特别是对抗补丁技术日益成熟,对交通标志识别构成现实威胁。防御技术呈现多元化发展趋势:主动防御如对抗训练从本质上提升模型鲁棒性;被动防御如输入重构和异常检测不改变模型即可过滤对抗样本;专用物理防御则结合算法与基础设施增强综合防护能力。

自动驾驶场景的动态环境特性和安全关键要求使其对抗防御面临独特挑战。云边端协同防御、多视角融合验证等创新方案针对性地解决了部分问题,但复杂环境下的防御可靠性仍需提升。实验评估表明,现有先进防御方案可将对抗攻击成功率降低80\%以上,但计算开销和泛化能力仍是瓶颈。

未来研究需在物理防御泛化性、轻量化架构、多模态融合、可解释性和标准化评估五大方向寻求突破。随着自动驾驶技术走向大规模应用,对抗攻防研究将从单纯的算法竞争转向系统级安全工程,需要跨学科协作构建更可靠的防护体系。只有确保视觉感知系统在对抗环境下的稳定性,自动驾驶技术才能真正实现安全落地应用。

\section{Modifications}

Modifying the template --- including but not limited to: adjusting
margins, typeface sizes, line spacing, paragraph and list definitions,
and the use of the \verb|\vspace| command to manually adjust the
vertical spacing between elements of your work --- is not allowed.

{\bfseries Your document will be returned to you for revision if
  modifications are discovered.}

\section{Typefaces}

The ``\verb|acmart|'' document class requires the use of the
``Libertine'' typeface family. Your \TeX\ installation should include
this set of packages. Please do not substitute other typefaces. The
``\verb|lmodern|'' and ``\verb|ltimes|'' packages should not be used,
as they will override the built-in typeface families.

\section{Title Information}

The title of your work should use capital letters appropriately -
\url{https://capitalizemytitle.com/} has useful rules for
capitalization. Use the {\verb|title|} command to define the title of
your work. If your work has a subtitle, define it with the
{\verb|subtitle|} command.  Do not insert line breaks in your title.

If your title is lengthy, you must define a short version to be used
in the page headers, to prevent overlapping text. The \verb|title|
command has a ``short title'' parameter:
\begin{verbatim}
  \title[short title]{full title}
\end{verbatim}

\section{Authors and Affiliations}

Each author must be defined separately for accurate metadata
identification. Multiple authors may share one affiliation. Authors'
names should not be abbreviated; use full first names wherever
possible. Include authors' e-mail addresses whenever possible.

Grouping authors' names or e-mail addresses, or providing an ``e-mail
alias,'' as shown below, is not acceptable:
\begin{verbatim}
  \author{Brooke Aster, David Mehldau}
  \email{dave,judy,steve@university.edu}
  \email{firstname.lastname@phillips.org}
\end{verbatim}

The \verb|authornote| and \verb|authornotemark| commands allow a note
to apply to multiple authors --- for example, if the first two authors
of an article contributed equally to the work.

If your author list is lengthy, you must define a shortened version of
the list of authors to be used in the page headers, to prevent
overlapping text. The following command should be placed just after
the last \verb|\author{}| definition:
\begin{verbatim}
  \renewcommand{\shortauthors}{McCartney, et al.}
\end{verbatim}
Omitting this command will force the use of a concatenated list of all
of the authors' names, which may result in overlapping text in the
page headers.

The article template's documentation, available at
\url{https://www.acm.org/publications/proceedings-template}, has a
complete explanation of these commands and tips for their effective
use.

Note that authors' addresses are mandatory for journal articles.

\section{Rights Information}

Authors of any work published by ACM will need to complete a rights
form. Depending on the kind of work, and the rights management choice
made by the author, this may be copyright transfer, permission,
license, or an OA (open access) agreement.

Regardless of the rights management choice, the author will receive a
copy of the completed rights form once it has been submitted. This
form contains \LaTeX\ commands that must be copied into the source
document. When the document source is compiled, these commands and
their parameters add formatted text to several areas of the final
document:
\begin{itemize}
\item the ``ACM Reference Format'' text on the first page.
\item the ``rights management'' text on the first page.
\item the conference information in the page header(s).
\end{itemize}

Rights information is unique to the work; if you are preparing several
works for an event, make sure to use the correct set of commands with
each of the works.

The ACM Reference Format text is required for all articles over one
page in length, and is optional for one-page articles (abstracts).

\section{CCS Concepts and User-Defined Keywords}

Two elements of the ``acmart'' document class provide powerful
taxonomic tools for you to help readers find your work in an online
search.

The ACM Computing Classification System ---
\url{https://www.acm.org/publications/class-2012} --- is a set of
classifiers and concepts that describe the computing
discipline. Authors can select entries from this classification
system, via \url{https://dl.acm.org/ccs/ccs.cfm}, and generate the
commands to be included in the \LaTeX\ source.

User-defined keywords are a comma-separated list of words and phrases
of the authors' choosing, providing a more flexible way of describing
the research being presented.

CCS concepts and user-defined keywords are required for for all
articles over two pages in length, and are optional for one- and
two-page articles (or abstracts).

\section{Sectioning Commands}

Your work should use standard \LaTeX\ sectioning commands:
\verb|section|, \verb|subsection|, \verb|subsubsection|, and
\verb|paragraph|. They should be numbered; do not remove the numbering
from the commands.

Simulating a sectioning command by setting the first word or words of
a paragraph in boldface or italicized text is {\bfseries not allowed.}

\section{Tables}

The ``\verb|acmart|'' document class includes the ``\verb|booktabs|''
package --- \url{https://ctan.org/pkg/booktabs} --- for preparing
high-quality tables.

Table captions are placed {\itshape above} the table.

Because tables cannot be split across pages, the best placement for
them is typically the top of the page nearest their initial cite.  To
ensure this proper ``floating'' placement of tables, use the
environment \textbf{table} to enclose the table's contents and the
table caption.  The contents of the table itself must go in the
\textbf{tabular} environment, to be aligned properly in rows and
columns, with the desired horizontal and vertical rules.  Again,
detailed instructions on \textbf{tabular} material are found in the
\textit{\LaTeX\ User's Guide}.

Immediately following this sentence is the point at which
Table~\ref{tab:freq} is included in the input file; compare the
placement of the table here with the table in the printed output of
this document.

\begin{table}
  \caption{Frequency of Special Characters}
  \label{tab:freq}
  \begin{tabular}{ccl}
    \toprule
    Non-English or Math&Frequency&Comments\\
    \midrule
    \O & 1 in 1,000& For Swedish names\\
    $\pi$ & 1 in 5& Common in math\\
    \$ & 4 in 5 & Used in business\\
    $\Psi^2_1$ & 1 in 40,000& Unexplained usage\\
  \bottomrule
\end{tabular}
\end{table}

To set a wider table, which takes up the whole width of the page's
live area, use the environment \textbf{table*} to enclose the table's
contents and the table caption.  As with a single-column table, this
wide table will ``float'' to a location deemed more
desirable. Immediately following this sentence is the point at which
Table~\ref{tab:commands} is included in the input file; again, it is
instructive to compare the placement of the table here with the table
in the printed output of this document.

\begin{table*}
  \caption{Some Typical Commands}
  \label{tab:commands}
  \begin{tabular}{ccl}
    \toprule
    Command &A Number & Comments\\
    \midrule
    \texttt{{\char'134}author} & 100& Author \\
    \texttt{{\char'134}table}& 300 & For tables\\
    \texttt{{\char'134}table*}& 400& For wider tables\\
    \bottomrule
  \end{tabular}
\end{table*}

\section{Math Equations}
You may want to display math equations in three distinct styles:
inline, numbered or non-numbered display.  Each of the three are
discussed in the next sections.

\subsection{Inline (In-text) Equations}
A formula that appears in the running text is called an inline or
in-text formula.  It is produced by the \textbf{math} environment,
which can be invoked with the usual
\texttt{{\char'134}begin\,\ldots{\char'134}end} construction or with
the short form \texttt{\$\,\ldots\$}. You can use any of the symbols
and structures, from $\alpha$ to $\omega$, available in
\LaTeX~\cite{Lamport:LaTeX}; this section will simply show a few
examples of in-text equations in context. Notice how this equation:
\begin{math}
  \lim_{n\rightarrow \infty}x=0
\end{math},
set here in in-line math style, looks slightly different when
set in display style.  (See next section).

\subsection{Display Equations}
A numbered display equation---one set off by vertical space from the
text and centered horizontally---is produced by the \textbf{equation}
environment. An unnumbered display equation is produced by the
\textbf{displaymath} environment.

Again, in either environment, you can use any of the symbols and
structures available in \LaTeX\@; this section will just give a couple
of examples of display equations in context.  First, consider the
equation, shown as an inline equation above:
\begin{equation}
  \lim_{n\rightarrow \infty}x=0
\end{equation}
Notice how it is formatted somewhat differently in
the \textbf{displaymath}
environment.  Now, we'll enter an unnumbered equation:
\begin{displaymath}
  \sum_{i=0}^{\infty} x + 1
\end{displaymath}
and follow it with another numbered equation:
\begin{equation}
  \sum_{i=0}^{\infty}x_i=\int_{0}^{\pi+2} f
\end{equation}
just to demonstrate \LaTeX's able handling of numbering.

\section{Figures}

The ``\verb|figure|'' environment should be used for figures. One or
more images can be placed within a figure. If your figure contains
third-party material, you must clearly identify it as such, as shown
in the example below.
\begin{figure}[h]
  \centering
  \includegraphics[width=\linewidth]{sample-franklin}
  \caption{1907 Franklin Model D roadster. Photograph by Harris \&
    Ewing, Inc. [Public domain], via Wikimedia
    Commons. (\url{https://goo.gl/VLCRBB}).}
  \Description{The 1907 Franklin Model D roadster.}
\end{figure}

Your figures should contain a caption which describes the figure to
the reader. Figure captions go below the figure. Your figures should
{\bfseries also} include a description suitable for screen readers, to
assist the visually-challenged to better understand your work.

Figure captions are placed {\itshape below} the figure.

\subsection{The ``Teaser Figure''}

A ``teaser figure'' is an image, or set of images in one figure, that
are placed after all author and affiliation information, and before
the body of the article, spanning the page. If you wish to have such a
figure in your article, place the command immediately before the
\verb|\maketitle| command:
\begin{verbatim}
  \begin{teaserfigure}
    \includegraphics[width=\textwidth]{sampleteaser}
    \caption{figure caption}
    \Description{figure description}
  \end{teaserfigure}
\end{verbatim}

\section{Citations and Bibliographies}

The use of \BibTeX\ for the preparation and formatting of one's
references is strongly recommended. Authors' names should be complete
--- use full first names (``Donald E. Knuth'') not initials
(``D. E. Knuth'') --- and the salient identifying features of a
reference should be included: title, year, volume, number, pages,
article DOI, etc.

The bibliography is included in your source document with these two
commands, placed just before the \verb|\end{document}| command:
\begin{verbatim}
  \bibliographystyle{ACM-Reference-Format}
  \bibliography{bibfile}
\end{verbatim}
where ``\verb|bibfile|'' is the name, without the ``\verb|.bib|''
suffix, of the \BibTeX\ file.

Citations and references are numbered by default. A small number of
ACM publications have citations and references formatted in the
``author year'' style; for these exceptions, please include this
command in the {\bfseries preamble} (before the command
``\verb|\begin{document}|'') of your \LaTeX\ source:
\begin{verbatim}
  \citestyle{acmauthoryear}
\end{verbatim}

  Some examples.  A paginated journal article \cite{Abril07}, an
  enumerated journal article \cite{Cohen07}, a reference to an entire
  issue \cite{JCohen96}, a monograph (whole book) \cite{Kosiur01}, a
  monograph/whole book in a series (see 2a in spec. document)
  \cite{Harel79}, a divisible-book such as an anthology or compilation
  \cite{Editor00} followed by the same example, however we only output
  the series if the volume number is given \cite{Editor00a} (so
  Editor00a's series should NOT be present since it has no vol. no.),
  a chapter in a divisible book \cite{Spector90}, a chapter in a
  divisible book in a series \cite{Douglass98}, a multi-volume work as
  book \cite{Knuth97}, an article in a proceedings (of a conference,
  symposium, workshop for example) (paginated proceedings article)
  \cite{Andler79}, a proceedings article with all possible elements
  \cite{Smith10}, an example of an enumerated proceedings article
  \cite{VanGundy07}, an informally published work \cite{Harel78},
  a couple of preprints \cite{Bornmann2019, AnzarootPBM14},
  a doctoral dissertation \cite{Clarkson85}, a master's thesis:
  \cite{anisi03}, an online document / world wide web resource
  \cite{Thornburg01, Ablamowicz07, Poker06}, a video game (Case 1)
  \cite{Obama08} and (Case 2) \cite{Novak03} and \cite{Lee05} and
  (Case 3) a patent \cite{JoeScientist001}, work accepted for
  publication \cite{rous08}, 'YYYYb'-test for prolific author
  \cite{SaeediMEJ10} and \cite{SaeediJETC10}. Other cites might
  contain 'duplicate' DOI and URLs (some SIAM articles)
  \cite{Kirschmer:2010:AEI:1958016.1958018}. Boris / Barbara Beeton:
  multi-volume works as books \cite{MR781536} and \cite{MR781537}. A
  couple of citations with DOIs:
  \cite{2004:ITE:1009386.1010128,Kirschmer:2010:AEI:1958016.1958018}. Online
  citations: \cite{TUGInstmem, Thornburg01, CTANacmart}. Artifacts:
  \cite{R} and \cite{UMassCitations}.

\section{Acknowledgments}

Identification of funding sources and other support, and thanks to
individuals and groups that assisted in the research and the
preparation of the work should be included in an acknowledgment
section, which is placed just before the reference section in your
document.

This section has a special environment:
\begin{verbatim}
  \begin{acks}
  ...
  \end{acks}
\end{verbatim}
so that the information contained therein can be more easily collected
during the article metadata extraction phase, and to ensure
consistency in the spelling of the section heading.

Authors should not prepare this section as a numbered or unnumbered {\verb|\section|}; please use the ``{\verb|acks|}'' environment.

\section{Appendices}

If your work needs an appendix, add it before the
``\verb|\end{document}|'' command at the conclusion of your source
document.

Start the appendix with the ``\verb|appendix|'' command:
\begin{verbatim}
  \appendix
\end{verbatim}
and note that in the appendix, sections are lettered, not
numbered. This document has two appendices, demonstrating the section
and subsection identification method.

\section{SIGCHI Extended Abstracts}

The ``\verb|sigchi-a|'' template style (available only in \LaTeX\ and
not in Word) produces a landscape-orientation formatted article, with
a wide left margin. Three environments are available for use with the
``\verb|sigchi-a|'' template style, and produce formatted output in
the margin:
\begin{itemize}
\item {\verb|sidebar|}:  Place formatted text in the margin.
\item {\verb|marginfigure|}: Place a figure in the margin.
\item {\verb|margintable|}: Place a table in the margin.
\end{itemize}

%%
%% The acknowledgments section is defined using the "acks" environment
%% (and NOT an unnumbered section). This ensures the proper
%% identification of the section in the article metadata, and the
%% consistent spelling of the heading.
\begin{acks}
To Robert, for the bagels and explaining CMYK and color spaces.
\end{acks}

%%
%% The next two lines define the bibliography style to be used, and
%% the bibliography file.
\bibliographystyle{ACM-Reference-Format}
\bibliography{sample-base}

%%
%% If your work has an appendix, this is the place to put it.
\appendix

\section{Research Methods}

\subsection{Part One}

Lorem ipsum dolor sit amet, consectetur adipiscing elit. Morbi
malesuada, quam in pulvinar varius, metus nunc fermentum urna, id
sollicitudin purus odio sit amet enim. Aliquam ullamcorper eu ipsum
vel mollis. Curabitur quis dictum nisl. Phasellus vel semper risus, et
lacinia dolor. Integer ultricies commodo sem nec semper.

\subsection{Part Two}

Etiam commodo feugiat nisl pulvinar pellentesque. Etiam auctor sodales
ligula, non varius nibh pulvinar semper. Suspendisse nec lectus non
ipsum convallis congue hendrerit vitae sapien. Donec at laoreet
eros. Vivamus non purus placerat, scelerisque diam eu, cursus
ante. Etiam aliquam tortor auctor efficitur mattis.

\section{Online Resources}

Nam id fermentum dui. Suspendisse sagittis tortor a nulla mollis, in
pulvinar ex pretium. Sed interdum orci quis metus euismod, et sagittis
enim maximus. Vestibulum gravida massa ut felis suscipit
congue. Quisque mattis elit a risus ultrices commodo venenatis eget
dui. Etiam sagittis eleifend elementum.

Nam interdum magna at lectus dignissim, ac dignissim lorem
rhoncus. Maecenas eu arcu ac neque placerat aliquam. Nunc pulvinar
massa et mattis lacinia.

\end{document}
\endinput
%%
%% End of file `sample-acmtog.tex'.
